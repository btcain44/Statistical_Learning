% Options for packages loaded elsewhere
\PassOptionsToPackage{unicode}{hyperref}
\PassOptionsToPackage{hyphens}{url}
%
\documentclass[
]{article}
\usepackage{amsmath,amssymb}
\usepackage{lmodern}
\usepackage{ifxetex,ifluatex}
\ifnum 0\ifxetex 1\fi\ifluatex 1\fi=0 % if pdftex
  \usepackage[T1]{fontenc}
  \usepackage[utf8]{inputenc}
  \usepackage{textcomp} % provide euro and other symbols
\else % if luatex or xetex
  \usepackage{unicode-math}
  \defaultfontfeatures{Scale=MatchLowercase}
  \defaultfontfeatures[\rmfamily]{Ligatures=TeX,Scale=1}
\fi
% Use upquote if available, for straight quotes in verbatim environments
\IfFileExists{upquote.sty}{\usepackage{upquote}}{}
\IfFileExists{microtype.sty}{% use microtype if available
  \usepackage[]{microtype}
  \UseMicrotypeSet[protrusion]{basicmath} % disable protrusion for tt fonts
}{}
\makeatletter
\@ifundefined{KOMAClassName}{% if non-KOMA class
  \IfFileExists{parskip.sty}{%
    \usepackage{parskip}
  }{% else
    \setlength{\parindent}{0pt}
    \setlength{\parskip}{6pt plus 2pt minus 1pt}}
}{% if KOMA class
  \KOMAoptions{parskip=half}}
\makeatother
\usepackage{xcolor}
\IfFileExists{xurl.sty}{\usepackage{xurl}}{} % add URL line breaks if available
\IfFileExists{bookmark.sty}{\usepackage{bookmark}}{\usepackage{hyperref}}
\hypersetup{
  pdftitle={Distribution\_Creation},
  hidelinks,
  pdfcreator={LaTeX via pandoc}}
\urlstyle{same} % disable monospaced font for URLs
\usepackage[margin=1in]{geometry}
\usepackage{color}
\usepackage{fancyvrb}
\newcommand{\VerbBar}{|}
\newcommand{\VERB}{\Verb[commandchars=\\\{\}]}
\DefineVerbatimEnvironment{Highlighting}{Verbatim}{commandchars=\\\{\}}
% Add ',fontsize=\small' for more characters per line
\usepackage{framed}
\definecolor{shadecolor}{RGB}{248,248,248}
\newenvironment{Shaded}{\begin{snugshade}}{\end{snugshade}}
\newcommand{\AlertTok}[1]{\textcolor[rgb]{0.94,0.16,0.16}{#1}}
\newcommand{\AnnotationTok}[1]{\textcolor[rgb]{0.56,0.35,0.01}{\textbf{\textit{#1}}}}
\newcommand{\AttributeTok}[1]{\textcolor[rgb]{0.77,0.63,0.00}{#1}}
\newcommand{\BaseNTok}[1]{\textcolor[rgb]{0.00,0.00,0.81}{#1}}
\newcommand{\BuiltInTok}[1]{#1}
\newcommand{\CharTok}[1]{\textcolor[rgb]{0.31,0.60,0.02}{#1}}
\newcommand{\CommentTok}[1]{\textcolor[rgb]{0.56,0.35,0.01}{\textit{#1}}}
\newcommand{\CommentVarTok}[1]{\textcolor[rgb]{0.56,0.35,0.01}{\textbf{\textit{#1}}}}
\newcommand{\ConstantTok}[1]{\textcolor[rgb]{0.00,0.00,0.00}{#1}}
\newcommand{\ControlFlowTok}[1]{\textcolor[rgb]{0.13,0.29,0.53}{\textbf{#1}}}
\newcommand{\DataTypeTok}[1]{\textcolor[rgb]{0.13,0.29,0.53}{#1}}
\newcommand{\DecValTok}[1]{\textcolor[rgb]{0.00,0.00,0.81}{#1}}
\newcommand{\DocumentationTok}[1]{\textcolor[rgb]{0.56,0.35,0.01}{\textbf{\textit{#1}}}}
\newcommand{\ErrorTok}[1]{\textcolor[rgb]{0.64,0.00,0.00}{\textbf{#1}}}
\newcommand{\ExtensionTok}[1]{#1}
\newcommand{\FloatTok}[1]{\textcolor[rgb]{0.00,0.00,0.81}{#1}}
\newcommand{\FunctionTok}[1]{\textcolor[rgb]{0.00,0.00,0.00}{#1}}
\newcommand{\ImportTok}[1]{#1}
\newcommand{\InformationTok}[1]{\textcolor[rgb]{0.56,0.35,0.01}{\textbf{\textit{#1}}}}
\newcommand{\KeywordTok}[1]{\textcolor[rgb]{0.13,0.29,0.53}{\textbf{#1}}}
\newcommand{\NormalTok}[1]{#1}
\newcommand{\OperatorTok}[1]{\textcolor[rgb]{0.81,0.36,0.00}{\textbf{#1}}}
\newcommand{\OtherTok}[1]{\textcolor[rgb]{0.56,0.35,0.01}{#1}}
\newcommand{\PreprocessorTok}[1]{\textcolor[rgb]{0.56,0.35,0.01}{\textit{#1}}}
\newcommand{\RegionMarkerTok}[1]{#1}
\newcommand{\SpecialCharTok}[1]{\textcolor[rgb]{0.00,0.00,0.00}{#1}}
\newcommand{\SpecialStringTok}[1]{\textcolor[rgb]{0.31,0.60,0.02}{#1}}
\newcommand{\StringTok}[1]{\textcolor[rgb]{0.31,0.60,0.02}{#1}}
\newcommand{\VariableTok}[1]{\textcolor[rgb]{0.00,0.00,0.00}{#1}}
\newcommand{\VerbatimStringTok}[1]{\textcolor[rgb]{0.31,0.60,0.02}{#1}}
\newcommand{\WarningTok}[1]{\textcolor[rgb]{0.56,0.35,0.01}{\textbf{\textit{#1}}}}
\usepackage{graphicx}
\makeatletter
\def\maxwidth{\ifdim\Gin@nat@width>\linewidth\linewidth\else\Gin@nat@width\fi}
\def\maxheight{\ifdim\Gin@nat@height>\textheight\textheight\else\Gin@nat@height\fi}
\makeatother
% Scale images if necessary, so that they will not overflow the page
% margins by default, and it is still possible to overwrite the defaults
% using explicit options in \includegraphics[width, height, ...]{}
\setkeys{Gin}{width=\maxwidth,height=\maxheight,keepaspectratio}
% Set default figure placement to htbp
\makeatletter
\def\fps@figure{htbp}
\makeatother
\setlength{\emergencystretch}{3em} % prevent overfull lines
\providecommand{\tightlist}{%
  \setlength{\itemsep}{0pt}\setlength{\parskip}{0pt}}
\setcounter{secnumdepth}{-\maxdimen} % remove section numbering
\ifluatex
  \usepackage{selnolig}  % disable illegal ligatures
\fi

\title{Distribution\_Creation}
\author{}
\date{\vspace{-2.5em}}

\begin{document}
\maketitle

\begin{Shaded}
\begin{Highlighting}[]
\FunctionTok{library}\NormalTok{(tinytex)}

\DocumentationTok{\#\#Set parameters}
\NormalTok{b\_0 }\OtherTok{=} \DecValTok{3}
\NormalTok{b\_1 }\OtherTok{=} \DecValTok{2}
\NormalTok{sigma\_2 }\OtherTok{=} \DecValTok{2}
\NormalTok{n }\OtherTok{=} \DecValTok{300}
\end{Highlighting}
\end{Shaded}

\hypertarget{a1-data-generation}{%
\subsection{A1 Data Generation}\label{a1-data-generation}}

The following graph displays the dataset generated under A1, which
implies error terms are iid normally distributed with \(\mu = 0\) and
and \(\sigma^2=2\) in our case.

\begin{Shaded}
\begin{Highlighting}[]
\DocumentationTok{\#\#Set seed for dataset generation}
\FunctionTok{set.seed}\NormalTok{(}\DecValTok{1}\NormalTok{)}

\DocumentationTok{\#\#Create the distribution}
\NormalTok{x }\OtherTok{=} \FunctionTok{sort}\NormalTok{(}\FunctionTok{runif}\NormalTok{(n,}\AttributeTok{min=}\DecValTok{0}\NormalTok{,}\AttributeTok{max=}\DecValTok{5}\NormalTok{))}
\NormalTok{e }\OtherTok{=} \FunctionTok{rnorm}\NormalTok{(n,}\DecValTok{0}\NormalTok{,}\FunctionTok{sqrt}\NormalTok{(sigma\_2))}
\NormalTok{y }\OtherTok{=}\NormalTok{ b\_0 }\SpecialCharTok{+}\NormalTok{ b\_1}\SpecialCharTok{*}\NormalTok{x }\SpecialCharTok{+}\NormalTok{ e}

\DocumentationTok{\#\#Plot the distribution }
\FunctionTok{plot}\NormalTok{(y}\SpecialCharTok{\textasciitilde{}}\NormalTok{x,}\AttributeTok{main=}\StringTok{\textquotesingle{}A1 Generated Data\textquotesingle{}}\NormalTok{,}\AttributeTok{xlab=}\StringTok{\textquotesingle{}X\textquotesingle{}}\NormalTok{,}\AttributeTok{ylab=}\StringTok{\textquotesingle{}Y\textquotesingle{}}\NormalTok{)}
\FunctionTok{abline}\NormalTok{(b\_0,b\_1,}\AttributeTok{col=}\StringTok{\textquotesingle{}red\textquotesingle{}}\NormalTok{)}
\end{Highlighting}
\end{Shaded}

\includegraphics{Distribution_Creation_files/figure-latex/A1 Distribution-1.pdf}

\#\#A2 Data Generation The next graph plots a dataset generated under
A2, where the assumption that \(\epsilon_i\) follow a normal
distribution are relaxed, and rather just requires \(\epsilon_i\) and
\(\epsilon_j\) are iid with \(Var[\epsilon_i]=\sigma^2\) and
\(E[\epsilon_i]=0\).

To generate this data, we instead draw random errors from a uniform
distribution, parameterized in the following way:
\(f(\epsilon) = \frac{1}{\beta}\cdot I_{(\alpha,\alpha+\beta)}(\epsilon)\),
with \(\alpha = -\frac{\sqrt{24}}{2}\) and \(\beta = \sqrt{24}\). This
will result in \(E[\epsilon_i]=0\) and \(Var[\epsilon_i]=\sigma^2=2\).

\begin{Shaded}
\begin{Highlighting}[]
\DocumentationTok{\#\#Set seed for dataset generation}
\FunctionTok{set.seed}\NormalTok{(}\DecValTok{1}\NormalTok{)}

\DocumentationTok{\#\#Create the distribution }
\NormalTok{x }\OtherTok{=} \FunctionTok{sort}\NormalTok{(}\FunctionTok{runif}\NormalTok{(n,}\AttributeTok{min=}\DecValTok{0}\NormalTok{,}\AttributeTok{max=}\DecValTok{5}\NormalTok{))}
\NormalTok{e }\OtherTok{=} \FunctionTok{runif}\NormalTok{(n, }\AttributeTok{min =} \SpecialCharTok{{-}}\FunctionTok{sqrt}\NormalTok{(}\DecValTok{24}\NormalTok{)}\SpecialCharTok{/}\DecValTok{2}\NormalTok{, }\AttributeTok{max =} \SpecialCharTok{{-}}\FunctionTok{sqrt}\NormalTok{(}\DecValTok{24}\NormalTok{)}\SpecialCharTok{/}\DecValTok{2}\SpecialCharTok{+}\FunctionTok{sqrt}\NormalTok{(}\DecValTok{24}\NormalTok{))}
\NormalTok{y }\OtherTok{=}\NormalTok{ b\_0 }\SpecialCharTok{+}\NormalTok{ b\_1}\SpecialCharTok{*}\NormalTok{x }\SpecialCharTok{+}\NormalTok{ e}

\DocumentationTok{\#\#Generate the plot}
\FunctionTok{plot}\NormalTok{(y}\SpecialCharTok{\textasciitilde{}}\NormalTok{x,}\AttributeTok{main=}\StringTok{\textquotesingle{}A2 Generated Data\textquotesingle{}}\NormalTok{,}\AttributeTok{xlab=}\StringTok{\textquotesingle{}X\textquotesingle{}}\NormalTok{,}\AttributeTok{ylab=}\StringTok{\textquotesingle{}Y\textquotesingle{}}\NormalTok{)}
\FunctionTok{abline}\NormalTok{(b\_0,b\_1,}\AttributeTok{col=}\StringTok{\textquotesingle{}red\textquotesingle{}}\NormalTok{)}
\end{Highlighting}
\end{Shaded}

\includegraphics{Distribution_Creation_files/figure-latex/A2 Distribution-1.pdf}

\hypertarget{a3-data-generation}{%
\subsection{A3 Data Generation}\label{a3-data-generation}}

This graph below plots the generated dataset under A3, which relaxes the
iid assumption previously required under A1 and A2. We must also
preserve that \(E[\epsilon_i]=0\) and \(Var[\epsilon_i]=\sigma^2=2\) and
that \(\epsilon_i\) and \(\epislon_j\) are uncorrelated for
\(i \neq j\). To do this, we create two vectors of random errors each
containing 150 observations, one vector generates \(\epsilon_i\) from
the uniform distribution above with \(E[\epsilon_i]=0\) and
\(Var[\epsilon_i]=2\) and the other vector generates \(\epsilon_i\) from
a normal distribution with \(\mu=0\) and \(\sigma^2=2\).

\begin{Shaded}
\begin{Highlighting}[]
\DocumentationTok{\#\#Set the seed for this module}
\FunctionTok{set.seed}\NormalTok{(}\DecValTok{1}\NormalTok{)}

\DocumentationTok{\#\#We will relax the iid assumption and have our random error come from two different}
\DocumentationTok{\#\#distributions, a normal and uniform distribution}
\NormalTok{x }\OtherTok{=} \FunctionTok{sort}\NormalTok{(}\FunctionTok{runif}\NormalTok{(n,}\AttributeTok{min=}\DecValTok{0}\NormalTok{,}\AttributeTok{max=}\DecValTok{5}\NormalTok{))}
\NormalTok{e1 }\OtherTok{=} \FunctionTok{runif}\NormalTok{(n}\SpecialCharTok{/}\DecValTok{2}\NormalTok{, }\AttributeTok{min =} \SpecialCharTok{{-}}\FunctionTok{sqrt}\NormalTok{(}\DecValTok{24}\NormalTok{)}\SpecialCharTok{/}\DecValTok{2}\NormalTok{, }\AttributeTok{max =} \SpecialCharTok{{-}}\FunctionTok{sqrt}\NormalTok{(}\DecValTok{24}\NormalTok{)}\SpecialCharTok{/}\DecValTok{2}\SpecialCharTok{+}\FunctionTok{sqrt}\NormalTok{(}\DecValTok{24}\NormalTok{))}
\NormalTok{e2 }\OtherTok{=} \FunctionTok{rnorm}\NormalTok{(n}\SpecialCharTok{/}\DecValTok{2}\NormalTok{,}\DecValTok{0}\NormalTok{,}\FunctionTok{sqrt}\NormalTok{(sigma\_2))}
\NormalTok{e }\OtherTok{=} \FunctionTok{c}\NormalTok{(e1,e2)}
\NormalTok{y }\OtherTok{=}\NormalTok{ b\_0 }\SpecialCharTok{+}\NormalTok{ b\_1}\SpecialCharTok{*}\NormalTok{x }\SpecialCharTok{+}\NormalTok{ e}

\DocumentationTok{\#\#Generate the plot}
\FunctionTok{plot}\NormalTok{(y}\SpecialCharTok{\textasciitilde{}}\NormalTok{x,}\AttributeTok{main=}\StringTok{\textquotesingle{}A3 Generated Data\textquotesingle{}}\NormalTok{,}\AttributeTok{xlab=}\StringTok{\textquotesingle{}X\textquotesingle{}}\NormalTok{,}\AttributeTok{ylab=}\StringTok{\textquotesingle{}Y\textquotesingle{}}\NormalTok{)}
\FunctionTok{abline}\NormalTok{(b\_0,b\_1,}\AttributeTok{col=}\StringTok{\textquotesingle{}red\textquotesingle{}}\NormalTok{)}
\end{Highlighting}
\end{Shaded}

\includegraphics{Distribution_Creation_files/figure-latex/A3 Distribution-1.pdf}

\hypertarget{interpretation}{%
\subsection{Interpretation:}\label{interpretation}}

One important thing to note is that for each vector of \(x\) values
generated, the range was from 0-5 as it became more difficult to
visually assess the differences in the plots over a larger range of x's.

In terms of visual differences, we can see in the A1 Generated Data
plot, that the points are very normally distributed across the true
population mean line and it is obvious that the residuals are normally
distributed. Moving to the A2 Generated Data chart, where the residuals
are iid on a uniform distribution, we can see that the data are spread
much more uniformly in relation to the true population mean line. We can
see this by directly comparing the A1 and A2 plots, in A1 we see a few
residuals that are straying away from the population mean line and might
be considered outliers, however in A2 we don't have points that are far
enough away from the rest of the points to be considered outliers. This
is likely because the residuals are uniformly distributed so we don't
observe outlier values like we do from the tails of a normal
distribution. Finally, in the A3 Generated data plot we can see how the
points switch from having uniformly distributed residuals to normally
distributed residuals around \(x=2.5\). Although the residuals are not
iid from one single distribution, they still come from two distributions
with the same variance of \(Var[\epsilon_i=]\sigma^2=2\) and where
\(E[\epsilon_i=0]\), thus satisfying A3, but not A1 or A2.

Note that the \texttt{echo\ =\ FALSE} parameter was added to the code
chunk to prevent printing of the R code that generated the plot.

\end{document}
